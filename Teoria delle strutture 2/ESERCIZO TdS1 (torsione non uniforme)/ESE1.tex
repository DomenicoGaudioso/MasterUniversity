\documentclass [a4paper,10pt]{article}
\usepackage[italian]{babel}
\usepackage[utf8]{inputenc}
\usepackage[T1]{fontenc}
\usepackage[top=1in, bottom=1in, left=1in, right=1in]{geometry}
\usepackage[T1]{fontenc}
\usepackage[scaled]{beramono}
\renewcommand*\familydefault{\ttdefault}
\usepackage{listings}
\usepackage{graphicx}

\usepackage{float} % serve per importare immagini





\lstset{
	language=Python,
	showstringspaces=false,
	formfeed=\newpage,
	tabsize=2,
	commentstyle=\itshape,
	basicstyle=\ttfamily,
	morekeywords={models, lambda, forms}
}

\newcommand{\code}[2]{
	\hrulefill
	\subsection*{#1}
	\lstinputlisting{#2}
	\vspace{2em}
}

\begin{document}
	
	
	%% Use this to include files
	\code{PROPIETA' DELLA SEZIONE }{Prop_sez.py}
	\newpage
	\code{SOLUZIONE LINEA ELASTICA PER LA FLESSIONE }{LE_Bending.py}
	\newpage
	\code{SOLUZIONE LINEA ELASTICA PER LA TORSIONE }{LE_Torsion.py}
	\newpage
	\code{CALCOLO DELLE TENSIONI }{Tension.py}
	\newpage
	\code{VALORI NUMERICI }{Numerico.py}
	
	\begin{figure}[H]
		\centering
		\includegraphics[scale=1,angle=90 ]{Cattura.png}
		\caption{VALORI}
		\label{valori}
	\end{figure}
	
	
\end{document}